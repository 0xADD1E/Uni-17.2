\documentclass{hitec}
\author{Addie Morrison}
\title{IME 254 Notes and Formulas}
\usepackage{hyperref}
\usepackage{graphicx}
\usepackage{amsmath}
\setlength{\parindent}{0pt}
\begin{document}
\maketitle
\subsubsection*{Probability Unions}
$P(A\cup B) = P(A) +P(B)-P(A\cap B)$

\subsubsection*{Compliments}
$1 = P(A)+P(A^C)$

\subsubsection*{Conditional Probability}
$P(A|B) = \dfrac{P(A\cap B)}{P(B)}$

\subsubsection*{Independent Events}
Events are independent if (and only if): $P(A|B) = P(A)$ (and $P(B|A) = P(B)$)

\subsubsection*{Multiplicative rule of Intersecting Events}
$P(A\cap B) =P(A|B)P(B) = P(B|A)P(A)$

$P(E) = P(E\cap A_1)+P(E\cap A_2)+...+P(E\cap A_k)$

\subsubsection*{Bayes' Rule}
$P(A|E) = \dfrac{P(E|A)P(A)}{P(E|A)P(A)+P(E|A^C)P(A^C)}$

\subsubsection*{Permutations}
Used to arrange $r$ items into distinct orders from a set of $n$ total distinct items.

$P^n_r = \dfrac{n!}{(n-r)!}$

\subsubsection*{Partitions}
To partition a set of $n$ items into $k$ sets with size $s_k$

$Partitions = \dfrac{n!}{s_1!\cdot s_2!\cdot s_k!}$

\subsubsection*{Combinations}
A special application of partitioning where $n$ elements are divided into two sets (size $s$ and $n-s$)

$C^n_s = \dfrac{n!}{(n-s)!\cdot s!}$
\subsubsection*{Pareto Diagram}
A histogram, ranked from highest to lwoest, with a line graph overlayed showing cumulative percentages.

\subsubsection*{Standard Deviation}
$\sigma = \sqrt{\dfrac{\Sigma^m_{i=1}(y_i - \overline{y})^2}{n-1}}$

At least 75\% of measurements will be withing $\overline{y}\pm 2\sigma$

At least 89\% of measurements will be withing $\overline{y}\pm 3\sigma$

\subsubsection*{Z-Score}
$z = \dfrac{y-\overline{y}}{\sigma}$

Data points are suspect outliars when $2\le |z| < 3$

Data points are highly suspect outliars when $|z| \ge 3$

\subsubsection*{Quartiles}
\begin{enumerate}
	\item Sort the dataset
	\item Calculate lower quartile bound $Q_L = \dfrac{1}{4(n+1)}$
	\item Calculate upper quartile bound $Q_U = \dfrac{3}{4(n+1)}$
	\item Everything else is in the middle quartile $Q_M$
\end{enumerate}
\end{document}
